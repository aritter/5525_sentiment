\documentclass[12pt, letterpaper]{article}
\usepackage[utf8]{inputenc}

\makeatletter
\newcommand{\@BIBLABEL}{\@emptybiblabel}
\newcommand{\@emptybiblabel}[1]{}
\makeatother
\usepackage[hidelinks]{hyperref}
 
\begin{document}
\title{CSE 5525 Homework 1: Text Classification}
\author{Alan Ritter}
\date{}
\maketitle
 
In this assignment you will implement the na\"{i}ve bayes algorithm for
sentiment classification.  You will train your models on a (provided) dataset of positive
and negative movie reviews and report prediction accuracy on a test set.

We provide you with starter Python code to help read in the data and evaluate the results of your model's predictions.
You are \emph{strongly} encouraged to make use of the provided code.  
If you prefer to implement everything from scratch, please talk to the instructor first.  Your submitted code
should run on the command line in a unix-like environment (e.g. Linux, OSX, Cygwin or Windows Supsystem for Linux).

Depending on the efficiency of your implementation the experiments required to complete the assignment may take some
time to run, so it is a good idea to start early.

\subsection*{Na\"{i}ve Bayes}
First, implement a na\"{i}ve bayes classifier.  The provided code in {\tt imdb.py} reads the data into a document-term matrix
using scipy's {\tt csr\_matrix} format (See \url{http://docs.scipy.org/doc/scipy-0.15.1/reference/generated/scipy.sparse.csr_matrix.html#scipy.sparse.csr_matrix} for details).
We recommend working with log-probabilities using addition instead of directly multiplying probabilities to avoid the possibility of floating point underflow (see: \url{https://en.wikipedia.org/wiki/List_of_logarithmic_identities}).

You can run the sample code like so:
\begin{verbatim}
python NaiveBayes.py aclImdb_small 1.0
\end{verbatim}

The two methods you will need to implement are {\tt NaiveBayes.Train} and {\tt NaiveBayes.Predict}.  Before you do this, the classifier always predicts +1 (positive).
Once you have implemented these methods, the code will print out accuracy.  Try running with different values of the smoothing hyperparameter ({\tt ALPHA}) (suggested values to try: 0.1, 0.5, 1.0, 10.0),
and record the results for your report.

\subsection*{What to Turn In}
Please turn in the following to the dropbox on Carmen:

\begin{enumerate}
  \item Your code
  \item A brief writeup that includes the numbers / evaluation requested above (text file format is fine).
\end{enumerate}

\end{document}
